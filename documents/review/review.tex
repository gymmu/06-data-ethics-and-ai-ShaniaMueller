\documentclass{article}

\usepackage[ngerman]{babel}
\usepackage[utf8]{inputenc}
\usepackage[T1]{fontenc}
\usepackage{hyperref}
\usepackage{csquotes}

\usepackage[
    backend=biber,
    style=apa,
    sortlocale=de_DE,
    natbib=true,
    url=false,
    doi=false,
    sortcites=true,
    sorting=nyt,
    isbn=false,
    hyperref=true,
    backref=false,
    giveninits=false,
    eprint=false]{biblatex}
\addbibresource{../references/bibliography.bib}

\title{Review des Papers "Ist es ethisch vertretbar, KI für die Erstellung von Kunst
zu verwenden?" von Noah Cooper}
\author{Shania Müller}
\date{\today}

\begin{document}
\maketitle



\section{Poitive Aspekte}

Noah hat in seiner Arbeit "Ist es ethisch vertretbar, KI für die Erstellung von Kunst zu verwenden?" viele gute Aspekte eingebracht.
Die Arbeit ist sehr übersichtlich und klar gegliedert. Jeder Abschnitt befasst sich mit einem bestimmten Teil des Themas, was es einfacher macht, 
alles zu verstehen. Diese Struktur hilft dem Leser, den Argumenten gut zu folgen.
Ein weiterer Pluspunkt ist, dass Noah viele verschiedene Quellen verwendet hat. Dies zeigt, dass er gründlich recherchiert hat und unterschiedliche Perspektiven berücksichtigte.
Besonders gut ist auch, dass Noah sowohl die positiven als auch die negativen Seiten der KI-Kunst beleuchtet. Er gibt allen die Möglichkeit, 
sich eine eigene Meinung zu bilden. Die Erklärungen zur Funktionsweise und zum Training der KI sind ebenfalls gut gelungen. Noah beschreibt klar und verständlich, wie KI trainiert wird 
und welche Schritte dabei nötig sind. Das hilft dem Leser, die technischen Aspekte besser zu verstehen und die Fähigkeiten von KI in der Kunst richtig einzuschätzen.

\section{Verbesserungsvorschläge}

Ich denke,es wäre spannend gewesen, wenn Noah mehr Bilder von KI-Kunst gezeigt hätte. Dadurch hätten wir besser verstanden, was die KI alles kann. 
Ausserdem hätte es interessant sein können, nicht nur die Leute zu hören, die KI-Kunst gut finden, sondern auch von denen zu hören, die dagegen sind.

\section{Fazit}

Noahs Arbeit hat uns besser verstehen lassen, wie KI in der Kunst eingesetzt wird. Er hat uns gezeigt, was dafür und dagegen spricht. Aber ich denke, 
es hätte noch besser sein können, wenn er mehr Beispiele gezeigt hätte. Dann hätten wir das Thema noch besser verstehen können.
Insgesamt ist es eine sehr gute Arbeit geworden!


\printbibliography

\end{document}
