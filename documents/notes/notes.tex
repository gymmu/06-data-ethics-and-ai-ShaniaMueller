\documentclass{article}

\usepackage[ngerman]{babel}
\usepackage[utf8]{inputenc}
\usepackage[T1]{fontenc}
\usepackage{hyperref}
\usepackage{csquotes}

\usepackage[
    backend=biber,
    style=apa,
    sortlocale=de_DE,
    natbib=true,
    url=false,
    doi=false,
    sortcites=true,
    sorting=nyt,
    isbn=false,
    hyperref=true,
    backref=false,
    giveninits=false,
    eprint=false]{biblatex}
\addbibresource{../references/bibliography.bib}

\title{Die KI in der Justiz}
\author{Shania Müller}
\date{\today}

\begin{document}
\maketitle

\abstract{
    Dieses Dokument ist eine Sammlung von Notizen zu dem Projekt. Die Struktur innerhalb des
    Projektes ist gleich ausgelegt wie in der Hauptarbeit, somit kann hier einfach geschrieben
    werden, und die Teile die man verwenden möchte, kann man direkt in die Hauptdatei ziehen.
}

\tableofcontents


\newpage
\section {Einleitung}

Welche Rolle spielt die KI in der Justiz?  
Die Zunehmende Verwendung der KI im Rechtssystem löst Diskussionen über den 
Umgang mit Daten und der Ethik aus. Während die KI-Technologie helfen kann, den
Gerichtsprozess zu vereinfachen, gibt neue Herausforderungen für Fairness und Datenschutz
In diesem Dokument schreibe ich daürber, wie Daten, Ethik und die KI in dem Justitzwesen
zusammenkommen und welche Bedeutung es für das Rechtssystem hat, und werde einen Fall vorstellen, 
bei dem ein Mann durch die KI vor Gericht musste. 

\section {COMPAS}

COMPAS 'Correctional Offender Management Profiling for Alternative Sanctions', ist ein 
algorithmisches Risikobewertungstool, das in der Strafjustiz verwendet wird, um die 
Wahrscheinlichkeit zu bewerten, dass ein Straftäter erneut straffällig wird.
Entwickelt von der Firma Northpointe, analysiert COMPAS verschiedene Faktoren, um 
einen Risikowert zu berechnen, der Richtern und anderen 
Strafjustizbehörden helfen soll, fundierte Entscheidungen über Strafen, Bewährung und andere 
Massnahmen zu treffen.

\section {KI-Training}

Wie wird die KI darauf trainiert? 
\newline +Zuerst werden umfangreiche Daten gesammelt, darunter Informationen zu früheren Verurteilungen, 
demographische Daten wie Alter und Geschlecht. Diese Daten werden bereinigt und in ein einheitliches 
Format gebracht, um sie analysieren zu können. 
\newline  + Dann beginnt der Trainingsprozess, bei dem maschinelle Lernalgorithmen Muster und Zusammenhänge in den 
Daten erkennen. Während dieses Prozesses wird das Modell ständig getestet und angepasst, um die 
Genauigkeit der Vorhersagen zu verbessern. 
\newline + Nach dem Training wird die KI in einer kontrollierten Umgebung weiterhin überwacht. Regelmässige 
Überprüfungen und Aktualisierungen stellen sicher, dass die Entscheidungen genau und fair bleiben. 
\newline + Ein wichtiger Aspekt ist die Transparenz und ethische Verantwortung. Entwickler müssen sicherstellen, 
dass die Algorithmen keine bestehenden Vorurteile verstärken und die Ergebnisse nachvollziehbar sind. 

\section {Verwendung vor Gericht}

\section {Menschen vor Gericht wegen KI}

Eric Loomis wurde vor Gericht gestellt, weil er im Februar 2013 in einem von ihm gestohlenen Auto 
erwischt wurde und wurde so wegen diversen Verkehrsdelikten angeklagt. Der Fall hatte eine besonders grosse 
Aufmerksamkeit erlangt, aufgrund der Art und Weise, wie für seine Verurteilung moderne
Technologien, wie die KI, eingesetzt wurden.  
Bei seiner Verurteilung nutzte das Gericht das Risikobewertungstool COMPAS, das ihn als hohes Risiko für erneute 
Straftaten einstufte. Loomis erhielt daraufhin eine sechseinhalbjährige Haftstrafe.
Loomis argumentierte, dass die Verwendung von COMPAS seine Rechte auf ein faires Verfahren verletzte. Er kritisierte 
die Intransparenz des Algorithmus, da weder er noch seine Anwälte die genaue Methode oder die dazugehörigen Daten 
nachvollziehen konnten. Er hatte Angst, dass der Algorithmus voreingenommen sein könnte und systematische Vorurteile 
gegen Minderheiten, in dem Fall ihn, hielt.




\section {Ethischer Zusammenhang}







\printbibliography

\end{document}
