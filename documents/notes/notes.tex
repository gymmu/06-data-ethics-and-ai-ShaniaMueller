\documentclass{article}

\usepackage[ngerman]{babel}
\usepackage[utf8]{inputenc}
\usepackage[T1]{fontenc}
\usepackage{hyperref}
\usepackage{csquotes}

\usepackage[
    backend=biber,
    style=apa,
    sortlocale=de_DE,
    natbib=true,
    url=false,
    doi=false,
    sortcites=true,
    sorting=nyt,
    isbn=false,
    hyperref=true,
    backref=false,
    giveninits=false,
    eprint=false]{biblatex}
\addbibresource{../references/bibliography.bib}

\title{Notizen zum Projekt Data Ethics}
\author{Shania Müller}
\date{\today}

\begin{document}
\maketitle

\abstract{
    Dieses Dokument ist eine Sammlung von Notizen zu dem Projekt. Die Struktur innerhalb des
    Projektes ist gleich ausgelegt wie in der Hauptarbeit, somit kann hier einfach geschrieben
    werden, und die Teile die man verwenden möchte, kann man direkt in die Hauptdatei ziehen.
}

\tableofcontents



\section {Notizen weil Baum}

Leitfrage: Welche Rolle spielt die KI in der Justiz?  

Einleitungsideen: 
- Die Zunehmende Verwendung der KI im Rechtssystem löst Diskussionen über den 
Umgang mit Daten und der Ethik aus. Während die KI-Technologie helfen kann, den
Gerichtsprozess zu vereinfachen, gibt neue Herausforderungen für Fairness und Datenschutz
In diesem Dokument schreibe ich daürber, wie Daten, Ethik und die KI in dem Justitzwesen
zusammenkommen und welche Bedeutung es für das Rechtssystem hat.

Aufbau: -Leitfrage -Einleitung -Menschen die wegen KI vor Gericht mussten -Verwendung der KI vom Gericht

\input{section_ai.tex}

\printbibliography

\end{document}
